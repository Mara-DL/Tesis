\begin{titlepage}
\hspace{-1.7cm} %Este comando es para mandar a la izquierda ;)
\begin{minipage}[t][0.03\textheight][c]{0.22\textwidth}
        \includegraphics[width=4.0cm]{FCFM-UNACH.png}
\end{minipage}\hspace{0.9cm}
\begin{minipage}[t][0.03\textheight][c]{0.69\textwidth}
\begin{center}
                \textsc{\huge Universidad Autónoma de Chiapas}\\[0.3cm]
                \hrule height 2.5pt
                \vspace{0.2cm}
                \hrule height1pt
                \vspace{0.3cm}
                \textsc{\Large Facultad de Ciencias en Física y Matemáticas}
\end{center}
\end{minipage}\hspace{0.2cm}
\begin{minipage}[t][0.03\textheight][c]{0.2\textwidth}
		\includegraphics[width=2.7cm]{logofcfm.png}
\end{minipage}\\
%%%%%%%%%%%%%%%%%%Aquí comienzan los otros dos minipages del título%%%%%%%%%%%%%%%%%%%%%%%%%%%%%%%%%%%%%%%%%%%%%%%%%%%%%%%%%%%%%%%%%%%%%%%%%%%%%%%%%%%%%%%%%%%%%%
\begin{minipage}[t][0.93\textheight][c]{0.06\textwidth}
\vspace{60pt}
    \begin{center}
        \vrule width1pt height18cm
        \vspace{5mm}
        \vrule width2.5pt height18cm
        \vspace{5mm}
        \vrule width1pt height18cm
   \end{center}
\end{minipage}\hspace{1.3cm} 
\begin{minipage}[t][0.95\textheight][c]{0.76\textwidth}

            \begin{center}
                {\Large\bfseries Estudio del impacto de la pandemia en un establecimiento de comida rápida: Análisis y estimación de parámetros para el modelo modificado de Cramér-Lundberg}\\[2cm]
                \textsc{\huge \textbf{T\, E\, S\, I\, S}}\\[1.5cm]
                \textsc{\large QUE PARA OBTENER EL TÍTULO DE:}\\[0.3cm]
                \textbf{\textsc{LICENCIADA EN MATEMÁTICAS APLICADAS}}\\[1.5cm]
                \textsc{\large PRESENTA:}\\[0.3cm]
                \textbf{\textsc{\large {MARA DOMINGUEZ LIMAS}}}\\[2cm]
                {\large\scshape Director de Tesis:\\[0.3cm]
                {\textbf{\large Dr. Yofre Hernán García Gómez }}}\\[2.0cm]
                \large{Tuxtla Gutiérrez, Chiapas a 07 de Marzo del 2025.}

            \end{center}
\end{minipage}
\end{titlepage}

\pagebreak[2]

\chapter*{Dedicatoria}
\begin{flushright}
\textit{A Dios por permitirme la vida y darme la sabiduría a lo largo de este proyecto. \\
 A mi familia, en particular a mi abuela, mi madre, mi tía y mi hermana por el apoyo incondicional, por tener la confianza en mi y ser parte de mi inspiración a seguir adelante.}
 \end{flushright}

\chapter*{Agradecimientos}

En primer lugar, deseo expresar mi más sincero agradecimiento y reconocimiento al Dr. Yofre Hernán García Gómez, director de esta tesis, por su invaluable guía y apoyo a lo largo de todo el proceso de investigación. Su experiencia, paciencia y compromiso fueron indispensables para la realización de este trabajo. Además, su acompañamiento emocional y el tiempo dedicado hicieron posible que adquiriera los conocimientos necesarios para el desarrollo de mi carrera profesional.

Extiendo también mi gratitud a los revisores, la Dra. María del Rosario Soler Zapata, el Dr. Armando Felipe Mendoza Pérez, el Dr. José Saúl Campos Orozco, y el Dr. Omar Antonio De La Cruz Curtois, por su disposición, dedicación y esfuerzo en la exhaustiva revisión de este trabajo. Sus observaciones y comentarios constructivos contribuyeron significativamente a alcanzar un nivel de excelencia en esta investigación.

De igual manera, agradezco profundamente a mis estimados profesores, especialmente al Dr. Orlando, Dr. Javier, la Mtra. Guadalupe, el Dr. Alfredo, la Dra. Karen, el Dr. Jesús, la Mtra. Greysi, el Dr. Mario Alberto, el Dr. Sergio, el Dr. Aarón, la Mtra. Marda, el Mtro. Eduardo, el Mtro. Elesban, la Mtra. Berenice, entre otros profesores de la facultad y personal administrativo. Junto con mi asesor de tesis y los revisores, me brindaron el apoyo académico y personal necesario para culminar mi carrera y llevar a cabo este proyecto de tesis. Su bondad, atención y generosidad, incluso al asumir responsabilidades más allá de la docencia, me ofrecieron valiosas lecciones de vida. Su compañía, consejos y enseñanzas fueron pilares fundamentales en este proceso.

Particularmente, quiero dirigir un especial agradecimiento al Ingeniero Omar Sivori Romero Vázquez, por confiar en mí y brindarme su apoyo de manera incondicional. Su generosidad y disposición para ayudarme en los momentos más difíciles son valores que siempre llevaré conmigo.

Asimismo, deseo expresar mi gratitud a mis queridos amigos Jennifer, Gustavo, Karla, Cristopher, Loranny, Jaziel, Osvaldo, Jordi, Anthony, Kevin, Jairo, Daniel, Ángel, Carlos, Andrés, Angie, Vasti, Rafael, Guadalupe, Enrique, Nely,  Karina y Araceli. Su apoyo, paciencia, cariño y amistad fueron esencial para la culminación de este proyecto. Agradezco profundamente cada momento compartido, su aliento constante y el conocimiento que generosamente compartieron conmigo. Su presencia ha sido un regalo invaluable en este camino.

Finalmente, mi mayor gratitud es para Dios y mi familia. A mi madre y mi abuela, aunque físicamente ya no estén, las siento siempre presentes en mi corazón. Gracias por ser mi mayor fuente de motivación e inculcarme el amor por el conocimiento. Las extraño cada día y les agradezco por el apoyo incondicional a lo largo de mi vida.

A mi tía, gracias por tu paciencia, amor, sacrificio y apoyo, que fueron fundamentales para que pudiera seguir adelante. Me brindaste todo lo necesario y más, permitiéndome avanzar con confianza. A mi hermana, gracias por creer siempre en mí y sentirte orgullosa incluso de mis pequeños logros. Ustedes son, y siempre serán, las personas más importantes en mi vida. Las amo profundamente, son mi orgullo y mi mayor inspiración para continuar superándome.

